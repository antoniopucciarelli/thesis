\chapter{Problem Framing}

This chapter aims to \textit{outline} the challenges that arise from designing a turbomachine using a \textbf{conventional approach}. 
Additionally, it introduces a new \textbf{potential design method} that can address some of these issues.

\section{Design Process}

% The study of a turbomachinery system involves a lot of \textit{effort} in terms of \textbf{modeling of the physics} that defines the problem and in 
% terms of the generation of the \textbf{best configuration} that fits the design constraints. 
% An efficient way to attack the problem is to \textbf{frame} it in terms of targets and constraints and to reduce the \textbf{complexity} of the 
% design process at its minimum in order to reduce the risks of \textit{bad design}.

Studying a turbomachinery system requires significant \textit{effort}, both in terms of modeling the underlying physics that define the problem 
and in generating the optimal configuration that aligns with design constraints. An efficient approach to \textbf{tackling} this challenge involves 
\textbf{framing} it in terms of targets and constraints. By minimizing the complexity of the design process, one can \textit{mitigate} the risks associated with poor design.

\subsection{Problem Classification}

An engineering problem can be described by \textbf{inputs} and \textbf{outputs}. 
An engineer should be able to get the best outputs giving the inputs in the less time possible. 
Inputs and outputs can be classified in order to better understand the problem and to solve it in the most effective way.

On the inputs side, they can be subdivided into: 
\begin{itemize}
    \item \textbf{Objectives}: these define the parameters that need optimization 
    \item \textbf{Constraints}: these define the region of interest for the study
\end{itemize}

On the outputs side, they can be classified as follows:
\begin{itemize}
    \item \textbf{Objectives}: the optimized parameters resulting from the design
    \item \textbf{Definitions}: the engineering representation of the outputs
\end{itemize}

A modeling framework serves as the link between the inputs and outputs of the problem.
A \textit{well-designed} framework allows reaching the best solution in reasonable time.

\begin{figure}
    \centering
    \newcommand\WIDTH{6cm}
\newcommand\HEIGHT{2cm}
\newcommand\Ydist{1cm}
\newcommand\XPOS{4cm}
\newcommand\TOTheight{6cm}
\newcommand\TOTwidth{\WIDTH}
\newcommand\PTS{2pt}
\newcommand\SHADOWgray{25}
\newcommand\PTSin{1pt}
\newcommand\TRANSPval{0.3}
\newcommand\BLOCKwidth{4cm}
\newcommand\BLOCKheight{3cm}
\newcommand\bigR{3cm}
\newcommand\medR{2cm}
\newcommand\lowR{1cm}
\newcommand\boxW{4.5cm}
\newcommand\boxH{4.3cm}
\newcommand\leftS{0.5cm}
\newcommand\physicsDist{0.35cm}
\newcommand\HspaceVal{-0.3cm}

\resizebox{\textwidth}{!}{
\begin{tikzpicture}
    
    \node[draw,
        rectangle,
        % rounded corners,
        line width = \PTS,
        minimum height = \TOTheight,
        minimum width = \TOTwidth
    ] (designProcess) at (0, 0) {};

    \node[draw,
        rectangle,
        % rounded corners,
        line width = \PTS,
        minimum height = 1cm,
        minimum width = \TOTwidth,
        above = -\PTS of designProcess.north
    ] (designTitle) {\textbf{Framework}};

    \node[draw,
        circle, 
        % opacity=\TRANSPval,
        line width = \PTS,
        minimum size = \bigR,
        fill = gray!\SHADOWgray,
        right = \physicsDist of designProcess.west
    ] (physics) {\textbf{Physics}};

    \node[draw,
        circle, 
        line width = \PTS,
        % opacity=\TRANSPval,
        minimum size = \medR,
        fill = gray!\SHADOWgray,
        above right = 0.5cm and 0.5cm of physics.east
    ] (modeling) {\textbf{Modeling}};

    \node[draw,
        circle, 
        % opacity=\TRANSPval,
        line width = \PTS,
        minimum size = \lowR,
        fill = gray!\SHADOWgray,
        below = 1cm of modeling.south
    ] (testing) {\textbf{Testing}};

    \coordinate[left=\leftS of designProcess.west]  (c1) {};
    \coordinate[above=2.5cm of c1]                 (c2) {};
    \coordinate[below=2.5cm of c1]                 (c3) {};
    \coordinate[right=\leftS of designProcess.east] (c4) {};
    \coordinate[above=2.5cm of c4]                 (c5) {};
    \coordinate[below=2.5cm of c4]                 (c6) {};

    \node[draw,
        rectangle,
        % opacity=\TRANSPval,
        minimum width = \BLOCKwidth, 
        minimum height = \BLOCKheight, 
        line width = \PTSin,
        left = \leftS of c2
    ] (obj) {\parbox[t][\boxH][c]{\boxW}{
        \small{\textbf{Objectives}
        \begin{itemize}
            \item Loss: $Y_p$
            \item Weight
        \end{itemize}  
    }}};

    \node[draw,
        rectangle,
        % opacity=\TRANSPval,
        minimum width = \BLOCKwidth, 
        minimum height = \BLOCKheight,
        line width = \PTSin,
        left = \leftS of c3
    ] (constr) {\parbox[t][\boxH][c]{\boxW}{
        \small{\textbf{Constraints}
        \begin{itemize}
            \item Tuning: $\alpha_1$ \& $\alpha_2$
            \item Flow: $Re$ \& $M_2$
            \item Annlus: $\frac{r_2}{r_1}$
            \item Manufacturing: $\frac{R_{TE}}{c}$
        \end{itemize}  
    }}};

    \node[draw,
        rectangle,
        % opacity=\TRANSPval,
        minimum width = \BLOCKwidth, 
        minimum height = \BLOCKheight,
        line width = \PTSin,
        right = \leftS of c5
    ] (obj1) {\parbox[t][\boxH][c]{\boxW}{
        \small{\textbf{Objectives}
        \begin{itemize}
            \item Achieved loss: $Y_p$
            \item Achieved weight
        \end{itemize}  
    }}};

    \node[draw,
        rectangle,
        % opacity=\TRANSPval,
        minimum width = \BLOCKwidth, 
        minimum height = \BLOCKheight,
        line width = \PTSin,
        right = \leftS of c6
    ] (def) {\parbox[t][\boxH][c]{\boxW}{
        \small{\textbf{Definition}
        \begin{itemize}
            \item Blade geometry
        \end{itemize}  
    }}};
    
    \draw[-latex, line width = 2pt] (physics)  to (modeling.west);
    \draw[-latex, line width = 2pt] (modeling) to (testing.north);
    \draw[-latex, line width = 2pt] (testing)  to (physics.south east);
    \draw[-latex, line width = 1pt] (obj.east)    -- (c2) -- (c1) to (designProcess.west);
    \draw[-latex, line width = 1pt] (constr.east) -- (c3) -- (c1) to (designProcess.west);
    \draw[-latex, line width = 1pt] (designProcess.east) -- (c4) -- (c5) to (obj1.west);
    \draw[-latex, line width = 1pt] (designProcess.east) -- (c4) -- (c6) to (def.west);

\end{tikzpicture}
}
    \caption{Turbomachinery design classification.}
    \label{fig:IO}
\end{figure}

Figure~\ref{fig:IO} represents a possible design problem.

\subsection{Turbomachinery Analysis in Complex Machines}

% A general modeling approach consists in decomposing the design of a complex machine in different \textbf{sub-problems} 
% which have a role in the performance of the whole system.
% 
% As example, it is possible to subdivide the study of an aeronautical engine as the study of its \textbf{main components} 
% - i.e. fan, compressor, combustion chamber, turbine, spool, gears, control systems and the other important assembles.
% 
% The main \textbf{macroscopic properties} of the system can be extracted from a \textbf{first analysis} of its main components.
% Focusing on the \textbf{turbine}; it is possible to estimate the \textbf{number of stages} and have 
% a first \textit{raw approximation} of the \textbf{flow deflection angles} in each stage.
% Once generated this preliminary design, it is necessary to draw \textbf{blades} which statisfy these properties. 
% The blade, on its side, will be \textbf{defined} by its component \textbf{sections}. The general properties of the blade will be 
% then sum up by its \textbf{mid-section}. As result, starting from the design of a turbomachinery, 
% the study will always be reconducted to the \textbf{design of the blades' mid-section}.
% 
% This result is the \textbf{pivoting point} of a \textit{new section design method}.

A general modeling approach involves decomposing the design of a complex machine into various \textbf{sub-problems}, 
each of which plays a role in the overall system's performance. For instance, the study of an aeronautical engine 
can be subdivided into the examination of its primary components: the fan, compressor, combustion chamber, turbine, spool, gears, control systems, 
and other important assemblies.

The principal \textbf{macroscopic properties} of the system can be derived from an initial analysis of 
its primary components. To illustrate, when focusing on the \textbf{turbine}, it becomes feasible to 
estimate the number of stages and to make a preliminary approximation of the flow deflection angles in each stage. 
After this preliminary design is formulated, the next step is to design blades that satisfy these specified properties. 
Each blade, in turn, will be characterized by its \textbf{constituent sections}. The collective properties of the blade will 
then be \textit{summarized} by its \textbf{mid-section}. Consequently, the design process for turbomachinery \textbf{invariably} leads back to 
the design of the blades' mid-section.

This outcome serves as the \textit{focal point} of a new section design methodology. 

Figure~\ref{fig:decomposition} depicts the central feature of problem decomposition.

\begin{figure}[!h]
    \centering
    \newcommand\UPPERWIDTH{6em}
\newcommand\MIDWIDTH{3.5em}
\newcommand\WIDTH{2.5em}
\newcommand\SHADOW{30}

\resizebox{\textwidth}{!}{
\begin{tikzpicture}[
    sibling distance=8.5em,
    every node/.style={
            shape=rectangle,
            draw,
            align=center,
            scale=1
        }
    ]

    \textbf{
    \node{Aeronautical engine}
        child[level distance=\UPPERWIDTH]{node{Fan}
            child[level distance=\MIDWIDTH]{node{Flow analysis}
                child[level distance=\WIDTH]{node{Blade}
                    child[level distance=\WIDTH]{node[fill=green!\SHADOW]{Section}}
            }}}
        child[level distance=\UPPERWIDTH]{node{Combustion\\chamber}
            child[level distance=\MIDWIDTH]{node{...}}}
        child[level distance=\UPPERWIDTH]{node{Compressor\\\&\\Turbine}
            child[level distance=\MIDWIDTH]{node{Flow analysis}
            child[level distance=\WIDTH]{node{Stage}
                child[level distance=\WIDTH]{node{Rotor}
                    child[level distance=\WIDTH]{node{Blade}
                        child[level distance=\WIDTH]{node[fill=green!\SHADOW]{Section}
                    }}}
                child[level distance=\WIDTH]{node{Stator}
                    child[level distance=\WIDTH]{node{Blade}
                        child[level distance=\WIDTH]{node[fill=green!\SHADOW]{Section}
                    }}}
            }}}
        child[level distance=\UPPERWIDTH]{node{Control\\systems}
            child[level distance=\MIDWIDTH]{node{...}}}
        child[level distance=\UPPERWIDTH]{node{Spools\\\&\\Gears}
            child[level distance=\MIDWIDTH]{node{...}}
    };
    }

\end{tikzpicture}
}
    \caption{Problem decomposition.}
    \label{fig:decomposition}
\end{figure}

\section{Design Methods}

% Having understood that the section generation is the \textbf{most repetive task} inside the turbomachinery design process, 
% it is convenient to understand also the state of the art for the section generation.
% There are different methods for the design of a blade section. Most of them can be defined as \textbf{iterative methods}, 
% where a \textbf{numerical scheme} is adopted for the \textit{description} of the flow properties around a geometry of study.  

% It is clear that \textbf{avoiding} this optimization task, which involves \textbf{CFD}, could play a \textit{key-role} in the \textbf{reduction} of the design time.

Recognizing that the section generation is the most \textbf{repetitive task} in the turbomachinery design process, 
it is advantageous to also \textbf{comprehend} the current \textbf{state of the art} in the section generation. 
Various methods exist for designing a blade section, many of which can be classified as \textbf{iterative processes}. 
In these methods, a \textbf{numerical scheme} is employed to describe flow properties around a chosen geometry.

It is evident that \textit{sidestepping} this optimization task, which \textbf{requires Computational Fluid Dynamics} (CFD), 
could substantially contribute to \textbf{reducing design time}.

\subsection{Iterative Based Method}

A further more classification can be made on the iterative based methods. 
These methods can be:

\begin{itemize}
    \item \textbf{Automated}: done using \textbf{computers}, where the solver tends \textit{blindly} to the optimal configuration.
    \item \textbf{Manaul}: done by \textbf{engineers}, where \textit{prior knowledge} and \textit{experience} allow reaching the optimum.
\end{itemize}

These methods all encounter the same challenge: \textbf{time is wasted on iterations involving CFD}~\cite{denton2010some}. 
On the other hand, each of these methods has its own \textbf{advantage}. In the case of \textbf{automated methods}, optimization is \textbf{expedited} through \textbf{computer usage}~\cite{shahpar2004review}. 
In contrast, \textbf{manual methods} leverage \textbf{prior knowledge} and \textbf{experience} to \textbf{crunch down the domain of study}, consequently reducing the time required for optimization.

\subsection{Non-Iterative Based Method}

The non-iterative-based method integrates \textbf{prior knowledge} and \textbf{experience}, utilizing data, along with the \textbf{speed} of computers through machine learning.

To describe this method, it is essential to introduce a \textit{new way} of design a blade.
Instead of generating a blade by \textit{iterating through various geometries}, 
this method constructs a blade \textit{using a dataset}~\cite{mitchell1997artificial} that represents a range of blade variations. 
Within this domain of blades, the optimal blade is determined by \textbf{maximizing the defined objectives} while adhering to the \textbf{imposed constraints}.

The \textbf{effectiveness} and \textbf{accuracy} of this method \textbf{rely} significantly on the \textbf{database} that must accurately represent the designated study domain. 
The sole optimization process employed by this method is associated with identifying appropriate blades that define the domain of study.

\subsection{Concepts}

After introducing this new method, its framework is \textit{grounded} in four other key concepts. 
These concepts dictate the approach to \textbf{generating} and \textbf{processing data} within the method. 
These concepts are:

\begin{itemize}
    \item The suitable machine learning model capable of \textit{accommodating} the \textbf{physics inherent in midsection generation}.
    \item The \textit{macroscopic} behavior of a blade.
    \item The \textit{local distribution of the load} across the blade.
    \item The \textit{dimensionality reduction} of the problem through a \textbf{parameterized representation} of both the blade and its loading.
\end{itemize}

\subsubsection{Physics \& Modeling}

% One of the main aspects to consider is the way data will be processed by the machine-learning algorithm. 
% A blade design is not subjected to external random noise which means that a blade will always \textit{behave} in
% the same way giving the same working conditions. This concept highlights the fact that there is a 
% \textit{one-to-one} correlation between the inputs and the geometry. 

% This concept is extremely important for the choice of the machine-learning model. 
% Given this data correlelation, a regression will be used for the interpolation of the 
% domain.

One of the primary aspects to consider is how data will be processed by the machine learning algorithm. Blade design is not susceptible to external random noise, implying that a blade will consistently exhibit the same behavior under identical operating conditions. This underscores the existence of a \textit{one-to-one} correlation between the inputs and the resulting geometry.

This concept holds great significance for selecting the appropriate machine learning model. Due to this data correlation, a regression approach will be employed for interpolating within the domain.

\subsubsection{Aerodynamic Duty}

Because of the high dimensionality of the problem, it is necessary to classify the objectives of study 
in order to minimize at its best the number of inputs. 

\begin{figure}[!h]

    \begin{center} 
    
        \begin{tikzpicture}
            \begin{axis}[
                width=0.5\textwidth, % Increased width
                axis equal,          % Set equal aspect ratio
                axis lines=none,     % Remove axis lines and labels
                xmin=-0.3, xmax=1.3, % Increased x limits
                ymin=-0.1, ymax=1.6,   % Increased y limits
            ]

            \addplot[black, line width=2pt] table[x index=0, y index=1, col sep=comma] {./pyFigure/csv/coords.csv};
            \addplot[black, line width=2pt] table[x index=2, y index=3, col sep=comma] {./pyFigure/csv/coords.csv};
            
            % Adding the second arrow with text
            \draw[-latex, line width=2.5pt] (axis cs:-0.25,0.55) -- node[below left] {$Re$} (axis cs:-0.05,0.35);
            \draw[-latex, line width=2.5pt] (axis cs:1.02,1) -- node[above left] {$M_2$} (axis cs:1.22,1.6);
            \draw[-, line width=1.0pt] (axis cs:-0.25,0.55) -- node[below] {$\alpha_1$} (axis cs:0.15, 0.55);
            \draw[-, line width=1.0pt] (axis cs:1.02,1) -- node[above left] {$\alpha_2$} (axis cs:1.45, 1);
            
            \end{axis}
        \end{tikzpicture}
    
    \end{center}

    \caption{Aerodynamic duty parameters.}
    \label{fig:aeroDuty}

\end{figure}

A way of describing the macroscopic working conditions of the flow is to define:

\begin{itemize}
    \item $\alpha_1$: inlet flow angle 
    \item $\alpha_2$: outlet flow angle 
    \item $M_2$: outlet Mach number 
    \item $Re$: Reynolds number of the flow
\end{itemize}

These four parameters, represented in Figure~\ref{fig:aeroDuty}, are the minimum requirements for the definition of a working condition or \textbf{duty}
of the blade. As result, these parameters will be indentified as \textbf{aerodynamic duty} of the blade.

\subsubsection{Aerodynamic Style}

% In the same manner, a blade can be described by its \textit{load behavior}. Contrarly to the \textbf{aerodynamic duty}
% the load can be defined as a local property of the blade. Following the same philosophy of the aerodynamic duty, 
% it is possible to \textbf{parametrize} the load around a blade in order to study the 
% \textbf{main load patterns} which are of \textbf{interest} in the engineering field. 

% % Figure~\ref{fig:loadPatters} shows the difference between a \textit{good} and \textit{applicable} load pattern and a 
% % \textit{poor-performing} pattern in the \textbf{high-pressure turbine} field.

% In the present study, only \textit{useful} \textbf{load-patterns} will be used for the generation of a database. 
% The usage of \textit{useful} \textbf{load-patterns} avoid \textit{unnecessary} diffusion and are the ones which
% generate the \textit{highest load differences} between suction and pressure side.   

% These \textbf{load-patterns} will take the name of \textbf{aerodynamic style} of the blade.


In a similar way, a blade's behavior can be characterized by its \textbf{load distribution}. In contrast to the \textbf{aerodynamic duty}, 
the load can be defined as a localized property of the blade. Adhering to the same philosophy as the aerodynamic duty, 
it is possible to \textbf{parametrize} the load distribution around a blade, thereby enabling the examination 
of key load patterns relevant to the engineering field.

Within the context of the current study, only \textit{pertinent load patterns} will be used for database generation. 
The usage of \textit{useful load-patterns} avoid \textit{unnecessary} diffusion and are the ones which
generate the \textit{highest load differences} between suction and pressure side.   

\subsubsection{Dimensionality}

% After having introduced two important classifications for the reduction of the \textbf{constraints dimensionality}, 
% a dimensionality reduction will be used as well for the definition of the blade. 

% The blade parametrization allows saving a lot of memory and sets a common definition of the whole possible blades inside the database. 

% Another important feature lies in reducing the complexity of data interpolation because of the parametrization:
% the interpolation of parameters is much easier than interpolating the \textit{behavior} of the blade's coordinates.

Having introduced two significant classifications for reducing constraint dimensionality, a dimensionality reduction technique will also be employed to define the blade.

Blade parametrization offers the advantage of \textbf{conserving memory} space and establishing a \textbf{uniform definition} for all potential blades within the database.

Furthermore, another noteworthy benefit emerges from the simplification of \textbf{data interpolation} due to parametrization: interpolating parameters is considerably simpler than interpolating the \textbf{intricate behavior of the blade's coordinates}.

\begin{figure}[!h]
    \centering
    \newcommand\WIDTH{3cm}
\newcommand\HEIGHT{0.8cm}
\newcommand\Ydist{0.2cm}
\newcommand\XPOS{1cm}
\newcommand\TOTheight{5.3cm}
\newcommand\TOTwidth{4cm}
\newcommand\PTS{1.2pt}
\newcommand\SHADOWred{50}
\newcommand\SHADOWgreen{30}
\newcommand\SHADOWblue{20}

\resizebox{\textwidth}{!}{
\begin{tikzpicture}
    
    \node[draw,
        rectangle,
        % rounded corners,
        line width = \PTS,
        minimum height = \TOTheight,
        minimum width = \TOTwidth
    ] (man) at (0, 0) {};

    \node[draw,
        rectangle,
        % rounded corners,
        line width = \PTS,
        minimum height = \TOTheight,
        minimum width = \TOTwidth,
        right = \XPOS of man
    ] (auto) {};

    \node[draw,
        rectangle,
        % rounded corners,
        line width = \PTS,
        minimum height = \TOTheight,
        minimum width = \TOTwidth,
        right = \XPOS of auto
    ] (ml) {};

    \node[draw, 
        rectangle,
        fill = blue!\SHADOWblue,
        line width = \PTS,
        minimum height = 0.8cm,
        above right = -\PTS and 0cm of man.north west
    ] (manTitle) {\textbf{\small{Manual}}};

    \node[draw, 
        rectangle,
        fill = blue!\SHADOWblue,
        line width = \PTS,
        minimum height = 0.8cm,
        above right = -\PTS and 0cm of auto.north west
    ] (autoTitle) {\textbf{\small{Automated}}};

    \node[draw, 
        rectangle,
        fill = blue!\SHADOWblue,
        line width = \PTS,
        minimum height = 0.8cm,
        above right = -\PTS and 0cm of ml.north west
    ] (mlTitle) {\textbf{\small{Machine learning}}};

    \node[draw,
        rectangle,
        rounded corners,
        line width = \PTS,
        minimum width = \WIDTH,
        minimum height = \HEIGHT, 
        % fill = red!\SHADOWred,
        below = \Ydist of man.north
    ] (manCFD) {\textbf{\small{Iterative CFD}}};

    \node[draw,
        rectangle,
        rounded corners,
        line width = \PTS,
        minimum width = \WIDTH,
        minimum height = \HEIGHT,
        below = \Ydist of manCFD
    ] (manObj) {\makecell[c]{\textbf{\small{Objectives \&}} \\ \textbf{\small{constraints}}}};%{\textbf{\small{Objectives \& constraints}}};

    \node[draw,
        rectangle,
        rounded corners,
        line width = \PTS,
        minimum width = \WIDTH,
        minimum height = \HEIGHT,
        below = \Ydist of manObj,
        fill = green!\SHADOWgreen
    ] (manKnow) {\makecell[c]{\textbf{\small{Using prior}} \\ \textbf{\small{knowledge}}}}; %{\textbf{Using prior knowledge}};

    \node[draw,
        rectangle,
        rounded corners,
        line width = \PTS,
        minimum width = \WIDTH,
        minimum height = \HEIGHT,
        below = \Ydist of manKnow,
        % fill = red!\SHADOWred
    ] (manMan) {\textbf{\small{Manual}}};

    \node[draw,
        rectangle,
        rounded corners,
        line width = \PTS,
        minimum width = \WIDTH,
        minimum height = \HEIGHT, 
        below = \Ydist of auto.north,
        % fill = red!\SHADOWred
    ] (autoCFD) {\textbf{\small{Iterative CFD}}};

    \node[draw,
        rectangle,
        rounded corners,
        line width = \PTS,
        minimum width = \WIDTH,
        minimum height = \HEIGHT,
        below = \Ydist of autoCFD
    ] (autoObj) {\makecell[c]{\textbf{\small{Objectives \&}} \\ \textbf{\small{constraints}}}};%{\textbf{\small{Objectives \& constraints}}};

    \node[draw,
        rectangle,
        rounded corners,
        line width = \PTS,
        minimum width = \WIDTH,
        minimum height = \HEIGHT,
        below = \Ydist of autoObj,
        % fill = red!\SHADOWred
    ] (autoKnow) {\makecell[c]{\textbf{\small{Not using prior}} \\ \textbf{\small{knowledge}}}};

    \node[draw,
        rectangle,
        rounded corners,
        line width = \PTS,
        minimum width = \WIDTH,
        minimum height = \HEIGHT,
        below = \Ydist of autoKnow,
        fill = green!\SHADOWgreen
    ] (autoAuto) {\textbf{\small{Automated}}}; 

    \node[draw,
        rectangle,
        rounded corners,
        line width = \PTS,
        minimum width = \WIDTH,
        minimum height = \HEIGHT, 
        below = \Ydist of ml.north,
        fill = green!\SHADOWgreen
    ] (mlCFD) {\textbf{\small{Regression}}};

    \node[draw,
        rectangle,
        rounded corners,
        line width = \PTS,
        minimum width = \WIDTH,
        minimum height = \HEIGHT,
        below = \Ydist of mlCFD,
        fill = green!\SHADOWgreen
    ] (mlObj) {\makecell[c]{\textbf{\small{Aerodynamic}} \\ \textbf{\small{duty \& style}}}};%{\textbf{\small{Objectives \& constraints}}};

    \node[draw,
        rectangle,
        rounded corners,
        line width = \PTS,
        minimum width = \WIDTH,
        minimum height = \HEIGHT,
        below = \Ydist of mlObj,
        fill = green!\SHADOWgreen
    ] (mlKnow) {\makecell[c]{\textbf{\small{Using prior}} \\ \textbf{\small{knowledge}}}};

    \node[draw,
        rectangle,
        rounded corners,
        line width = \PTS,
        minimum width = \WIDTH,
        minimum height = \HEIGHT,
        below = \Ydist of mlKnow,
        fill = green!\SHADOWgreen
    ] (mlAuto) {\textbf{\small{Automated}}}; 
 
\end{tikzpicture}
}
    \caption{Representation of the main differences between the method.} 
    % and the main characteristics of the new framework.}
    \label{fig:framework}
\end{figure}

