\chapter{Conclusions}

This chapter concludes the present work and highlights the main \textbf{results}, \textbf{limitations}, and possible \textbf{achievable improvements}.

\section{Introduction and Reiteration}

In this study, a groundbreaking design methodology has been developed for the field of turbomachinery blade design. This novel approach eliminates the need for conventional CFD simulations, significantly reducing design time while establishing insightful correlations between data and blade geometries.

\section{Summary of Findings}

The most crucial outcome of this research is the successful validation of the proposed methodology. It demonstrates applicability in blade design, offering speed and the ability to break down the design space into principal geometrical modes.

\section{Implication and Significance}

These identified modes present the opportunity to design blades more efficiently, employing a limited number of parameters compared to traditional blade parametrization methods. The establishment of a correlation between blade geometry and parametrized loading distribution further empowers the model. Simultaneously, the model defines the design's limitations in terms of loading, granting designers a comprehensive understanding of the design space and constraints in turbomachinery design.

\section{Limitations}

It's important to acknowledge that the model's boundaries are defined by the laws of physics. The sensitivity of database generation to loading parametrization highlights the close connection to physics. Moreover, the quality of the input database significantly impacts the quality of the resulting blades. Ensuring a high-quality database is essential for optimal results.

\section{Future Directions}

Future research directions could explore how blade geometry behaves with variations in the Reynolds number. Investigating blade geometry changes under diverse loading conditions also holds promise for advancing the methodology.

% \section{Final Reflections}

% The model's practicality in the industry is evident, serving as an initial design layer for blade generation. It offers a comprehensive range of macroscopic values that hold pivotal importance in the realm of turbomachinery.

\section{Closing Statements}

% The impact of this innovative methodology cannot be overstated. 
The model's practicality in the industry is evident, serving as an initial design layer for blade generation. By seamlessly integrating physics and machine learning through data, it opens up a new avenue for designing turbomachinery systems. This efficient tool accelerates the design process, making it accessible to designers across the field. The profound significance lies in its ability to reshape the design landscape, creating a faster and more intuitive approach for designing turbomachinery systems.