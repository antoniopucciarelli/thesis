\chapter{\texttt{datablade}}
\label{chapter:code}

This chapter introduces the \textbf{code} developed for the \textbf{database generation} and \textbf{analysis}. 
It will also explain the main aspects of the \textbf{code} in terms of \textbf{blade generation} and \textbf{optimization strategy}.
The code has been named \texttt{datablade}.

\section{Structure}

Figure~\ref{fig:databladeStructure} represents the directory subdivision of \texttt{datablade}. 
The program structure is based on blocks that can be combined for different purposes.  

\begin{figure}[!h]
  \tikzstyle{every node}=[draw=black,thick,anchor=west]
  \tikzstyle{selected}=[draw=red,fill=red!30]
  \tikzstyle{optional}=[dashed,fill=gray!50]
  \tikzstyle{plain}=[draw=none,anchor=west]
  \begin{minipage}{0.5\textwidth}
    \centering
    \begin{tikzpicture}[
      grow via three points={one child at (0.5,-0.7) and
      two children at (0.5,-0.7) and (0.5,-1.4)},
      edge from parent path={(\tikzparentnode.south) |- (\tikzchildnode.west)},
      scale=1]
      \texttt{
      \node {datablade}
        child { node {docs}}		
        child { node {misc}}
        child { node {module}}
        child { node [selected] {src}}
        child { node {test}}
        child { node [plain] {LICENSE}}
        child { node [plain] {setup.py}}
        child { node [plain] {setup.cfg}}
        child { node [plain] {README.md}};
      }
    \end{tikzpicture}
  \end{minipage}
  \begin{minipage}{0.5\textwidth}
    \centering
    \begin{tikzpicture}[
      font=\scriptsize,
      grow via three points={one child at (0.5,-0.7) and
      two children at (0.5,-0.7) and (0.5,-1.4)},
      edge from parent path={(\tikzparentnode.south) |- (\tikzchildnode.west)},
      scale=1]
      \texttt{
      \node [selected] {datablade/src}
        child { node {compileLIB}}
        child { node {databaseLIB}}
        child { node {execLIB}}
        child { node {kulfanLIB}}
        child { node {loadLIB}}
        child { node {misesLIB}}
        child { node {optimizationLIB}}
        child { node {performanceLIB}}
        child { node {postProcessingLIB}};
      }
    \end{tikzpicture}
  \end{minipage}
  \caption{\texttt{datablade} structure}
  \label{fig:databladeStructure}
\end{figure}

\section{Configuration File}

\texttt{datablade} optimization is configured using a \textbf{configuration file}. The configuration file is in \texttt{.json} format for allowing better \textbf{reading} and \textbf{handability}.

The following listings are parts of the configuration file - \texttt{config.json} - used in  \texttt{datablade}. All the dictionary entries are called \textit{as closer as possible} to the physical variables.

Listing~\ref{listing:configGuess} initilizes the initial guess for the blade optimization. Starting from its geometry - camberline, suction side and pressure side properties - and 
allocating the pitch and the trailing edge radius (which is kept constant at $R_{TE} = 1.25 \cdot 10^{-2}$ for the whole database).

\lstinputlisting[language=json, firstline=2, lastline=30, caption=\texttt{config.json} structure: initial guess ($\boldsymbol{x}_0$)., label=listing:configGuess]{./code/config.json.txt}

Listing~\ref{listing:configAero} defines the aerodynamic style and aerodynamic duty of the blade.

\lstinputlisting[language=json, firstline=31, lastline=44, caption=\texttt{config.json} structure: aerodynamic style and aerodynamic duty setup., label=listing:configAero]{./code/config.json.txt}

Listing~\ref{listing:configMises} sets the \texttt{MISES} simulation's entries and defines the parameters used in the cost function.

\lstinputlisting[language=json, firstline=57, lastline=72, caption=\texttt{config.json} structure: \texttt{MISES} configuration and cost function parameters., label=listing:configMises]{./code/config.json.txt}

At the end of \texttt{config.json}, in Listing~\ref{listing:configOpt}, there is the definition of the optimization strategy used in \texttt{datablade}.

\lstinputlisting[language=json, firstline=73, lastline=115, caption=\texttt{config.json} structure: optimization strategy setup., label=listing:configOpt]{./code/config.json.txt}

\section{Optimization}

The optimization algorithm used in \texttt{datablade} is a classic \textbf{gradient-free} method. 
The method used is the \textbf{Simplex} method~\cite{nelder1965simplex}. 
Even though the method finds the \textbf{local minima}, satisfactory results can be guaranteed using:

\begin{itemize}
    \item an appropriate choice of the initial guess, $\boldsymbol{x}_0$ 
    \item a \textbf{dimensionality adaptation strategy} for the problem
    \item an appropriate cost function which guarantees reliable results
\end{itemize}

\subsection{Dimensionality Adaptation}

% The blade optimization is a multi dimensional optimization problem which translates in a high cost in time and resources for 
% the computation of the optimum. Because of this problem, it is necessary to find a \textit{smart} way for the optimization of the blades. 
% The optimization strategy adapted in this work allows reaching good results for the generation of data that will be part of the database.
% This strategy consists in optimizing the blade varying the dimensionality of the problem. 

% The optimizer starts optimizing a blade parametrized with few parameters. 
% A low degree of freedom blade cannot generate a satisfactory result but, on the other hand, 
% it allows to flatten the design space avoiding \textit{bad design intervals} - in which normally the optimizer fails to converge.
% Once the low parameters optimization has converged, it is very likely that an increase in degree of freedom generates a better solution. 

% It has been tested that the best performance are reached doubling the degree of freedom of the blade parametrization after every optimization.

% If a blade converged with a low degree of freedom parametrization, the computed blade is scaled such that the database
% is populated by blade parametrized with the same number of parameters.

The optimization of blades poses a challenging multidimensional optimization problem, demanding significant time and resources to attain the optimal solution. Given this complexity, an \textit{intelligent approach} is necessary to streamline blade optimization. The strategy employed in this study demonstrates the ability to yield \textit{favorable outcomes} for constructing the database. This strategy involves optimizing blades by \textbf{manipulating the dimensionality of the problem}~\cite{clark2019step}.

The optimization process initiates by optimizing a blade with a \textbf{limited set of parameters}. While a blade with restricted degrees of freedom \textbf{may not produce an ideal outcome}, it does contribute to \textbf{flattening the design space}, thereby \textbf{mitigating unfavorable design intervals} where the optimizer might struggle to converge. Once optimization with fewer parameters achieves convergence, it's highly probable that increasing the degree of freedom will lead to \textbf{improved solutions}.

Empirical tests have revealed that the most favorable outcomes are achieved by \textbf{doubling} the degree of freedom in blade parametrization after each optimization cycle. Should a blade successfully converge with a low degree of freedom parametrization, the computed blade is subsequently scaled to populate the database with blades parametrized with the same number of parameters.

\subsection{Cost Function}

The blade optimization is related to two main errors: 

\begin{itemize}
    \item the \textbf{load error} over the blade, which is computed as the root mean squared error over the loading 
    \item the \textbf{flow exit angle error}
\end{itemize}

The root mean squared error is computed using Equation~(\ref{eqn:RMSE}).
The exit angle flow error is then computed using Equation~(\ref{eqn:angleError}). 

\begin{align}
    RMSE            & = \sqrt{\frac{1}{N_{points}} \cdot \sum_{i = 1}^{N_{points}} \Bigg( \frac{M_{real}}{M_{TE, real}} \Bigg|_{i} - \frac{M_{target}}{M_{TE, target}} \Bigg|_{i} \Bigg)^2} 
    \label{eqn:RMSE} \\ 
    \Delta \alpha_2 & = \Big| \alpha_{2, real} - \alpha_{2, target} \Big| 
    \label{eqn:angleError} 
\end{align}

Equation~(\ref{eqn:RMSE}) gets the Mach fraction distribution along the blade using 
the \texttt{.dat} file computed by \texttt{EDP} module in \texttt{MISES} and the Mach 
fraction distribution using the aerodynamic style curve computed in Chapter~\ref{chapter:framework}.

Once $RMSE$ and $\Delta \alpha_2$ have been computed, the cost function, Equation~(\ref{eqn:costFunction}), is a blend of those properties.
It is important that the cost function relates the $RMSE$ and $\Delta \alpha_2$ with a \textbf{product}; this in order
to reduce the \textit{competition} between the two variables during the optimization.

\begin{equation}
    cost = RMSE \cdot \Big[ 1 + 0.04 \cdot \Big( max \Big( 0, \ \Delta \alpha_2 - 1.0 \Big) \Big)^{2.0} \Big]
    \label{eqn:costFunction}
\end{equation}

Equation~\ref{eqn:costFunction} features the product between the $RMSE$ with another term which is a blend of different quantities.
The $0.04$ factor represents a scale which adapts the angle error to the overall cost function, it dictates how important the angle error, 
$\Delta \alpha_2$, should be compared to the $RMSE$. The scaling value acts on the terms in round parenteses. 
The value in the round parenteses is a threshold switch for the angle error. The threshold is set up by the 
$max(0, \ \Delta \alpha_2 - 1.0 )$ term. Once $ \Delta \alpha_2 - 1.0 > 0$, the switch activates and $\Delta \alpha_2$ contributes to the cost function.
On top of that, the \textit{switch} is \textbf{squared}; this in order to allow a \textbf{smooth transition of the error} from a region where \textit{just}
$RMSE$ contributes to the cost function - for $ \Delta \alpha_2 - 1.0 \leq 0$ - to the region where both $RMSE$ and $\Delta \alpha_2$ contribute to the overall cost function
- for $ \Delta \alpha_2 - 1.0 > 0$.

% The cost function has a double purpose. On one hand it generates a \textit{law} which 
% defines the convergence properties of the optimization algorithm. On the other hand 
% it provides a \textit{threshold of acceptability} for the optimized blades. If the value
% of the cost function for an optimized blade is high, this blade can be not taken into account 
% in order to have a more accurate database.

The cost function serves a dual purpose. Firstly, it establishes a \textit{guideline} governing the convergence behavior of 
the optimization algorithm. Secondly, it provides a \textit{threshold of acceptability} for the optimized blades. 
If the cost function value for an optimized blade is high, that particular blade might be excluded to ensure a more precise database.

\section{Optimizer}

On top of the optimization strategy there is the optimizer. 
The \texttt{datablade} optimizer combines the \texttt{MISES} software with the \texttt{scipy} module~\cite{2020SciPy-NMeth}.
These two components allow good performance and robustness. 

Figure~\ref{alg:datablade} illustrates how the blade is optimized by \texttt{datablade}.
It comprehends the optimization algorithm keeping the blade dimensionality fixed alongside 
the scaling process to increase convergence. Figure~\ref{alg:datablade} starts with the loading 
of the configuration file, \texttt{config.json}, which provides the initial guess - where the 
optimization starts -, the aerodynamic style and the aerodynamic duty of the blade. Once 
these properties are loaded, the program starts to optimize the blade using an
optimization based on the dimensionality adaptation of the blade to the problem. 


\begin{figure}[!h]

    \centering
    
    \resizebox{\textwidth}{!}{
    \begin{tikzpicture}[
        font=\fontsize{9}{9}\ttfamily, 
        thick, 
        align=center, 
        node distance=0.45cm, 
        remember picture, 
        x=\textwidth, 
        y=\textheight,
        every node/.style={
            align=center,
            scale=1
        }
    ]
            
    % Start block
    \node[draw,
        rectangle,
        rounded corners,
        % minimum width=2.5cm,
        % minimum height=0.2cm
        ] (start) {START};
    
    % getting guess of the solution 
    \node[draw,
        trapezium, 
        trapezium stretches=true, % A later addition
        trapezium left angle=70, 
        trapezium right angle=110, 
        text centered,
        below=of start,      
        % minimum width=3.0cm,  
        % minimum height=1cm,
    ] (aerodynamicDuty) {aerodynamic duty setup\\$\alpha_1$, $\alpha_2$, $M_2$ and $Re$};  

    % getting guess of the solution 
    \node[draw,
        trapezium, 
        trapezium stretches=true, % A later addition
        trapezium left angle=70, 
        trapezium right angle=110, 
        text centered,
        below=of aerodynamicDuty,      
        % minimum width=3.0cm,  
        % minimum height=1cm,
     ] (aerodynamicStyle) {aerodynamic style setup\\$\frac{M_P}{M_{TE}}$, $\frac{L_P}{L_{surf}}$, $\frac{M_{LE}}{M_{TE}}\frac{M_2}{M_1}$ and $\frac{M_{PS}}{M_{TE}}\frac{M_2}{M_{1, ax}}$}; 

    % getting guess of the solution 
    \node[draw,
        trapezium, 
        trapezium stretches=true, % A later addition
        trapezium left angle=70, 
        trapezium right angle=110, 
        text centered, 
        below=of aerodynamicStyle,      
        % minimum width=3.0cm,  
        % minimum height=1cm,
    ] (guess) {guessing $\boldsymbol{x}_{0}$\\from \texttt{config.json}};  

    % dimensionality check
    \node[draw, 
        diamond,
        below=of guess
      ] (dimCheck) {$N_{suct} \leq N_{suct_{target}}$\\[0.1cm]$N_{press} \leq N_{press_{target}}$};
    
    % setting up control point
    \coordinate[right=6cm of dimCheck] (controlPoint) {};

    % convergence check
    \node[draw, 
        diamond,
        above=3cm of controlPoint
    ] (convCheck) {$iter > iterTol$\\[0.1cm]$cost < costTol$\\[0.1cm]$\sigma < \sigma_{tol}$};

    % x optimization
    \node[draw, 
        below=0.5cm of convCheck,
        % minimum width=6.5cm,  
        % minimum height=1cm,
    ] (xUpdate) {$\boldsymbol{x}$ update};
  
    % coordinate computation
    \node[draw, 
        below=of xUpdate,
        % minimum width=6.5cm,  
        % minimum height=1cm,
    ] (settingPar) {blade coordinates computation using\\ Kulfan parametrization};
    
    % ISET input file generation
    \node[draw,
        below=of settingPar,
        % minimum width=5.5cm,
        % minimum height=1cm,
        ] (ISETsetup) {setting up \texttt{ISET} configuration file};
    
    % ISES input file generation
    \node[draw,
        below=of ISETsetup,
        % minimum width=5.5cm,
        % minimum height=1cm,
    ] (ISESsetup) {setting up \texttt{ISES} configuration file};

    % ISET input file generation
    \node[draw,
        below=of ISESsetup,
        % minimum width=5.5cm,
        % minimum height=1cm,
    ] (MISEScomputation) {running \texttt{MISES} solver: grid generator, \\flow solver \& flow post-processing};
        
    % cost computation
    \node[draw,
        below=of MISEScomputation,
        % minimum width=5.5cm,
        % minimum height=1cm,
    ] (costComputation) {computing cost function};

    % end 
    \node[draw,
        rectangle,
        rounded corners,
        left=1cm of dimCheck,
    ] (end) {END};

    % coordinates for arrotw
    \coordinate[left=1.7cm of convCheck] (c1) {};
    \coordinate[] (c2) at (c1 |- costComputation) {};
    \coordinate[right=1.5cm of convCheck] (c3) {};
    \coordinate[below=0.35cm of convCheck] (c4) {};
    \coordinate[right=2cm of dimCheck] (c5) {};
    \coordinate[] (c6) at (c5 |- convCheck.north) {};
    \coordinate[right=1.7cm of convCheck] (c7) {};
    \coordinate[below=0.35cm of costComputation] (c8) {};
    \coordinate[] (c9) at (c7 |- c8) {};
    \coordinate[] (c10) at (dimCheck |- c9) {};
    \coordinate[left=0.5cm of dimCheck] (c11) {};

    % arrow drawing
    \draw[-latex] (start) to (aerodynamicDuty);
    \draw[-latex] (aerodynamicDuty) to (aerodynamicStyle);
    \draw[-latex] (aerodynamicStyle) to (guess);
    \draw[-latex] (guess) to (dimCheck);
    \draw[-latex] (dimCheck.east) -- node[anchor=south] {yes} (c5) -- (c6) to (convCheck.north);
    \draw[-latex] (convCheck.south) -- node[anchor=east] {no} (c4) to (xUpdate.north);
    \draw[-latex] (xUpdate) to (settingPar);
    \draw[-latex] (settingPar) to (ISETsetup);
    \draw[-latex] (ISETsetup) to (ISESsetup);
    \draw[-latex] (ISESsetup) to (MISEScomputation);
    \draw[-latex] (MISEScomputation) to (costComputation);
    \draw[-latex] (costComputation.west) -- (c2) -- (c1) to (convCheck.west);
    \draw[-latex] (convCheck.east) -- node[anchor=north] {yes} (c3) -- (c7) -- (c9) -- (c10) -- (dimCheck.south); 
    \draw[-latex] (dimCheck.west) -- node[anchor=south] {no} (c11) -- (end.east);

    \end{tikzpicture}
    }
    
    \caption{\texttt{datablade} optimizer structure.}
    \label{alg:datablade}
    
\end{figure}

